\documentclass[12pt]{article}
\usepackage[utf8]{inputenc}
\usepackage[english]{babel}
\usepackage{amsmath}
\usepackage{amsfonts}
\usepackage{amssymb}
\usepackage{color}
\usepackage[left=2.00cm, right=2.00cm]{geometry}
\usepackage{graphicx}
\usepackage{natbib}
\topmargin 0.0cm
\oddsidemargin 0.5cm
\evensidemargin 0.5cm
\textwidth 16cm 
\textheight 22cm
\begin{document}
\begin{justify}
\linespread{2}
\hyphenation{const-ti-tu-tion-al arm-pit}
\title{Metacommunity dynamics of complex life-cycles in heavily-disturbed landscapes}
\maketitle
\author{{Rodrigo Riera${{^1}$,${^{*}}$, Joana Vasconcelos${{^1}{^2}{^3}}$, Ricardo Sousa ${{^2}{^4}}$,Carlos J. Melían${{^5}$
\\
\small{$^{1}$Departamento de Ecología, Facultad de Ciencias, Universidad Católica de la Santísima Concepci\´on, Concepción, Chile\\$^{2}$Research Service of the Regional Fisheries Department (DSI/DRP), Estrada da Pontinha, 9004-562 Funchal, Madeira, Portugal\\$^{3}$Marine and Environmental Sciences Centre (MARE), Quinta do Lorde Marina, Sítio da Piedade, 9200-044 Caniçal, Madeira, Portugal\\$^{4}\\$^{5}$Department of Fish Ecology and Evolution, EAWAG Center for Ecology, Evolution and Biogeochemistry, Switzerland\\ $^{*}$corresponding author: rriera@ucsc.cl
\\
\doublespacing
\section{Abstract}
Human sprawl is all over the globe, affecting most of species on Earth. Metacommunity models predicts that populations can survive in a fragmented landscape consisting
of several patches. Structured metacommunity models have been extensively used to determine dynamics of species in future environmental scenarios. We herein developed a metacommunity dynamics model for highly-fragmented species in areas with high anthropogenic pressure. Four models were developed considering the main trait-based drivers concerning mating, dispersal, biotic and environmental variables. 

\section{Keywords}
Metacommunities, dispersal dynamics, individual-based model, human-driven perturbations, Approximate Bayesian Computational

\section{Introduction}
Movement is a pervasive trait among species on Earth. All species disperse to some extent, because resources become limited as population grow, but a few have a worldwide distribution. Mobility occurs during the whole life cycle or limited to dispersal stages, such as seeds in plants or eggs in fish. Most of the species move over short distances and, to a lesser extent, to greater distances \citep{davidson2004quantifying}. Dispersal, defined as the movement of individuals away from their source \citep{nathan2003methods}, is a key strategy to increase fitness in dynamic landscapes by moving to different environments. Hence, ecological and evolutionary proccesses that occur in one location may drive changes at far sites or in other ecosystems through the ecological coupling of long-distance dispersal \citep{gaines2007connecting}. Fitness variability between habitat patches is the driving force for dispersal evolution, comprising pivotal ecological processes such as, interactions, habitat quality and competition (\citep{bowler2005causes}. Yet, highly-fragmented populations may not experience sufficient levels of dispersal, with a subsequent increase of inbreeding which lead to decreased fitness in many species \citep{fath2018encyclopedia}.Small populations with high dispersal rates may be prone to extinction under such situations  whilst marginal populations with immigration rates may experience a rescue effect where individuals may persist \citep{eriksson2014emergence}. Unfortunately, most models assumed a fixed dispersal strategy \citep{mccallum2001should, levin2003ecology}, but understanding the link between dispersal and population dynamics is pivotal for predicting population responses to habitat loss and fragmentation, and interaction with alien species \citep{bowler2005causes}.

Human footprint is everywhere, even in protected areas to safeguard biodiversity \citep{ tournadre2014anthropogenic, jones2018one}. Earth´s ecosystems are becoming increasingly fragmented, with extensive habitat loss and species depletion \citep{haddad2015habitat}. Fragmentation modifies the structure, diversity, dynamics, species composition and recruitment rates of communities through habitat reduction and subsequent edge effects \citep{laurance1998effects, short2011extinction}. Natural fragmentation has been associated with processes that maintain or even increase biodiversity \citep{tilman1988plant}, but human-driven fragmentation is a major threat for biodiversity \citep{pimm2000biodiversity} and a disruption of ecosystem processes \citep{achard2002determination}. The effects of fragmentation have been traditionally studied using the framework of island biogeography \citep{whittaker2007island}, but recent works have included tools, e.g. population genetics \citep{young1996population}, or approaches such as, metapopulation \citep{hanski2004ecology} and metacommunity theories \citep{mouquet2011extinction}.

Metapopulation models predict the survival of highly-disperse populations regardless local depletion of species \citep{hanski1999habitat,akccakaya2007role}. However, habitat loss and fragmentation may be so extensive that result in a massive species extinction \citep{montoya2008habitat,Rybicki and Hanksi,
2013,haddad2015habitat}. Model predictions have shown that a decrease in connectivity among assemblages, from continuous to sparsely-distributed populations is accompanied by species loss \citep{metzger2009time, niebuhr2015survival}. The role of natural (e.g. stochastic events) and human-driven (e.g. pollution, habitat loss and exploitation) perturbations have been extensively studied in the last decades (\citep{dornelas2010disturbance}and references therein) and it has been frequently studied in ecological theory \citep{volkov2007pattern, gardner2008spatial}. The degree of anthropogenic pressure may be a capital factor for landscape connectivity, since directly affects the persistence or decrease of assemblages \citep{supp2014species}. Besides perturbations, dispersal rates in disturbed landscapes need to be high in order to maintain viable populations \citep{provan2008high}. However, individual-based models (IBM) are needed to predict dispersal rates in species where small specimens,i.e. juveniles, are not reproductively active and even larger-sized adults harbour the highest reproductive potential \citep{hendriks2008scaling,  werner2011reproductive}. IBM simulate populations as being composed of discrete individual organisms \citep{deangelis2014individual, an2016predicting}. In IBMs the actions of single individuals are simulated and they interact with others and the landscape they live in \citep{deangelis2005individual}. They incorporate attributes vary among the individuals and can change through time such as, growth, foraging, dispersal and reproduction, among others \citep{martin2013predicting, deangelis2014individual}. These models have been used as size-structured methods to integrate a high variety of data which output are pivotal for management purposes \citep{punt2013review} and conservation strategies \citep{nabe2014effects}. Metapopulation models have been primarily used to certain vagile species such as insects \citep{harrison1995testing, hilker2006parameterizing}or mammals \citep{krohne1997dynamics, lawes2000patch}, but sessile or semi-sessile species have been neglected though a high proportion have a larval dispersal phase \citep{sale2006merging, gaggiotti2017metapopulations}. This strategy is extensively used in most of reduced-mobility marine species, with a subsequent increase of populations connectivity to nearby areas. However, assemblages from distant places are prone to isolation, as genetic analysis showed in semisessile mollusks from oceanic archipelagos (\citep{corte1996population, gaggiotti2017metapopulations, faria2017disentangling}. Oceanographic conditions, i.e. currents and eddies, and large geographic distances showed to be pivotal environmental factors to decrease the dispersal capacity e.g. \citep{palumbi1994genetic}. High larvae mortality plays also a crucial role, mainly due to predation and larvae traits, i.e. growth, reproductive and recruitment strategies, and mobility capacity \citep{cowen2000connectivity, cowen2009larval} in a limited connectivity within species. However, genetic analysis have been observed to be insufficient to accurately determine the demographic connectivity among populations of terrestrial \citep{chapuis2011challenges}, estuarine \citep{turner2002genetic} and marine species \citep{hawkins2016fisheries}.

We herein develop metacommunity models based on individuals of two intertidal species, namely the limpets \textit{Patella aspera} and \textit{P. candei}. The first model assumes that dispersal rates between patches are distance-dependent, with low rates between highly-separated assemblages. The second model assumes that dispersal rates are positively correlated to individual density. The third model assumes that larger individuals have higher reproductive potential. The fourth model considers a low probability of dispersal to peripheral assemblages relative to central ones. We confront the model with long-time series data (1994-2014) of two commercial limpet species (\textit{Patella candei} and \textit{P. aspera}) in two overpopulated oceanic islands ($>$ 500 inhab $km^{-2}$) with a high harvesting pressure \citep{riera2016clear, sousa2019long} for details). Former studies highlighted the sharp decrease of sizes of both limpets, a sympton of overexploitation and hence, a major driver for future local extinction of these species in Tenerife \citep{riera2016clear} and Madeira \citep{sousa2019long}. We herein develop a series of models  of metacommunity dynamics to predict future trends on the limpet populations of both islands. The predictive power of these models is pivotal to articulate an integrative management plan to guide future conservation actions. A deep understanding of the eco-evolutionary effects of connectivity loss is essential to guide future conservation actions (\citep{travis2013dispersal, urban2001landscape}. Trait evolution is a pivotal topic in metacommunity dynamics, and we herein particularly focused on traits related to mating, dispersal and biological and environmental traits. These traits greatly influence the dispersal potential of species, specifically important in highly-fragmented populations, and complex life-cycles with contrasting dispersal potential.
We expected individuals from size-biased populations to have lower dispersal rates than well-structured populations with a good representation of adults. Also, theory predicts that populations from more fragmented habitats are subjected to a lower propensity of dispersal.

\section{Material and Methods}
\subsection{Metapopulation dynamics}
Spatial population dynamics are based on eco-evolutionary traits of species. Metacommunity and metapopulation models (\underline{REFERENCES HEREIN, INCLUDE SEVERAL STATE-OF-THE-ART PAPERS} Xu et al. 2006 CHECK IT OUT!!) focused on simple life cycles (EXAMPLE) continuous habitats (EXAMPLES) or in equilibrium between extinctions and colonizations \citep{hanhanski1994practical}. Also, our metacommunity model is based on stochastic patch occupancy models \citep{moilanen2004spomsim} that have been extensively used in metapopulation studies \citep{ovaskainen2004individual}. However, species with complex life stages, with patchy distributions in a highly-fragmented landscape have been neglected by these models. Current scenarios show that an extensive number of species are subjected to intense human-driven disturbances directly, e.g. hunting or fishing, or indirectly, e.g. habitat loss and degradation. The connnectivity of populations is of utmost importance to prevent their extinction in local and global populations. Reproductive individuals become rather important for the maintenance of future generations, however, extraction activities are mostly primarily targeting larger and older specimens, with the highest reproductive potential.

\subsection {Trait dynamics}
Four main traits were included in the present metacommunity model, comprising the pivotal ecological, evolutionary and environmental factors concerning the studied metacommunities.

(i)\textit{MATING}
The studied species have external fecundation, with random encounter between gametes in the water column. Thus, a lottery model was assumed. A third submodel was conducted to integrate the exponential decay of gametes after their release and the subsequent decrease of fertilized eggs per time.

(ii)\textit{ABIOTIC}
We assumed that the study species are constrained to disperse passively by the large-scale (the Canary Current, \citep{barton1998transition} and meso-scale (eddies) oceanographic conditions. Current data were integrated in a submodel to calculate distances between sites. We also assumed that exposed sites are less affected by harvesting pressure that sites easily-accesible all over the year; thus, a submodel integrating exposition and extraction as trade-off was herein carried out.

(iii)\textit{BIOTIC}
Harvesting pressure is the main responsible of the high-fragmented limpet populations in both studied islands \citep{riera2016clear,sousa2019long}. We assumed that this anthropogenic driver remains constant throughout the last 20 years, and with no seasonal variations within a year. We herein used the limpet size as a reproductive proxy, since individuals over 3.5 mm are considered adults, since 50\% of individuals are reproductive \citep{henriques2012life}, and the higher the size the higher the reproductive potential \citep{boaventura2002analysis, martins2017exploitation}. Several submodels were conducted to integrate the human-driven consequences on the studied species. Management measures were integrated, i.e. closed season (December-March in Madeira); minimal harvesting size (40 mm in Madeira and 45 mm in Canaries for both species), harvesting limitations per fishermen (15 kg per day for professional fishermen in Madeira; 10 kg per day for professional fishermen and 3 kg per day for recreative fishermen). A submodel comprises the dependence of extraction on limpet size, i.e. that larger limpets are subjected to more intense harvesting pressure than shorter individuals.
No human-driven factors were also herein included in this model trait, since death probability regardless the individual size was also integrated in a submodel.

(iv)\textit{DISPERSAL}
The 5-stage life-cycle of the studied species greatly determines the dispersal potential of \textit{Patella candei} and \textit{P. aspera}. The limpet size is a proxy of reproductive potential, i.e. the larger the specimen the higher the number of gametes released by the individual. Because of the different dispersal potential of first stages, i.e. egg, trocophore and veliger larvae, three kinds of dispersal were considered in the submodel. Namely, global dispersion was considered for large distances ($>$ 100 kms), belonging to eggs and trocophores , regional dispersion for distances ranging from 1 to 100 km, belonging to trocophores and veligers, and local dispersion for shorter distance ($<$ 1 km), belonging to late-stage veligers. No dispersion was also included in the submodel to represent juveniles and adults, considered sessile organisms with homing behaviour.

\subsection{Case study}
Two species of limpets (\textit{Patella candei} and \textit{P. aspera}) were herein used as model system, since they harbour a series of traits that have been undercovered by former metacommunity models. They are characterized by a complex five-staged life cycle, with ontogenic differences in dispersal rate, from large-dispersed eggs to sessile adults. The transition among stages is fast, driven by post-fertilization hours.The size of these species are used as a reproductive proxy, the higher the size the higher the reproductive potential, with individuals over 35 mm considered adults since 0\percent of individuals are reproductive \citep{henriques2012life}. Hence, a decrease of limpet size is a sympton of local extinctions in patches with low connectivity. Lastly, limpets have been heavily exploited in the study locations, underpinning patchy-dispersed populations within a disturbed landscape. Metacommunity dynamics may reveal the main traits structuring the dispersal, recruitment and settlement of these species in these isolated areas, without input from extensive coastal continental areas.

\subsection{Modelling framework}
A metacommunity model was developed to explore the probability of occupancy of the two studied species in each oceanic island, i.e. Madeira and Tenerife, based on previous spatially-explicit metacommunity models \citep{hanski1999habitat,hanski2000metapopulation, ovaskainen2001spatially, hanski2002extinction, bertuzzo2015metapopulation}.In the present model, each pixel of the modeled landscape is assumed to be a patch that may be either occupied or not by larvae of the studied limpet species. We assumed that the study species are constrained to disperse passively by the large-scale (the Canary Current, \citep{barton1998transition} and meso-scale (eddies) oceanographic conditions, and also by the active dispersion of the larvae \citep{henriques2012life}. Former studies have demonstrated that the main driver of the sharp decrease of limpet populations is the human harvesting pressure, we assumed that remains constant throughout the last 20 years, and with no seasonal variations within a year. Both species (\textit{Patella candei} and \textit{P. aspera}) are in clear regression in both islands due to their commercial interest \citep{riera2016clear, sousa2019long}.

We consider an empirically sampled landscape consisting of N sites. At each site, there are two competing consumer species each with a complex life-cycle population of different abundances. Resources and natural predators are not considered in the present metacommunity model. To model spatio-temporal changes in population abundances, trait, population and dispersal dynamics need to be defined (see Table 1). We combine the lottery competition model \citep{chesson1981environmental} with trait-based dynamics accounting for complex life-cycles, exposition to human settlements, Marine Protected Areas (MPAs), spatial heterogeneity and explotation intensity. The spatial structure of the present model is static, and the site matrix comprises 3 vectors, i.e. size, exposition to environment and extraction by humans. Individuals are dynamic spatial and temporally, because of their dispersal potential.

Model 1: Infinite sites in homogeneous landscapes
This scenario represents a large number of sites with no differentiation between North and South, different rates of human-driven disturbance or asymmetry in dispersal probabilities. Details of the biotic (weak competition), abiotic, mating and dispersal traits

Model 2: Infinite sites in heterogeneous landscapes 
A large number of sites with differentitation between North and South, exposition to human settlements and asymmetry in dispersal probabilities
Details of the biotic (weak competition), abiotic, mating and dispersal traits

Model 3: Finite sites in homogeneous landscapes 
Details of the biotic (strong competition), abiotic, mating and dispersal traits

Model 4: Finite sites in heterogeneous landscapes 
Details of the biotic (strong competition), abiotic, mating and dispersal traits

\section{Results}

(3) OUTPUTS
IN-OUT ratio -> Source or Sink sites
Overlap with Marine Protected Areas (MPAs), Sources-Sinks?

(4) Approximate Bayesian Computational (ABC) and Bayes Networks (BN)

\section{Discussion}

% IDEAS TO EXPLORE, THAT NEED TO BE POINTED OUT AS MAINPOINTS IN THE DISCUSSION
Hihgly fragmented populations
Complex life-history (5 phases), with different dispersal rates
Trait-based model,based on empirical data
Exploring gradient of complexity in the model

%IDEAS CONCERNING LIMPETS AND OTHER INTERTIDAL MOLLUSKS
This strategy is extensively used in most of reduced-mobility marine species, with a subsequent increase of populations connectivity to nearby areas. However, assemblages from distant places are prone to isolation, as genetic analysis showed in semisessile mollusks from oceanic archipelagos (Corte-Real et al. 1996; Bird et al. 2011; Faria et al. 2017). Oceanographic conditions, i.e. currents and eddies, and large geographic distances showed to be pivotal environmental factors to decrease the dispersal capacity (e.g. Palumbi et al. 1994). High larvae mortality plays also a crucial role, mainly due to predation and larvae traits, i.e. growth, reproductive and recruitment strategies, and mobility capacity (Cowen et al. 2000; Cowen and Sponaugle 2009) in a limited connectivity within species. However, genetic analysis have been observed to be insufficient to accurately determine the demographic connectivity among populations of terrestrial (Chapuis et al. 2011), estuarine (Turner et al. 2002) and marine species (Hawkins et al. 2016).


\section{Acknowledgements}




\bibliographystyle{evolution}
\bibliography{BiblioLimpets}
%\insertbibliography


\end{document}



