\documentclass[12pt]{article}
\color(blanco)
\begin{document}
\hyphenation{const-ti-tu-tion-al arm-pit}
\title{Metapopulation dynamics of complex life-cycles in heavily-disturbed landscapes: Insights from exploited species in oceanic archipelagos}
\maketitle

\section{Abstract}
The dynamics of sparse populations are of utmost importance in ecology. Structured metapopulation models have been extensively used to determine extinction thresholds and populations persistence in future environmental scenarios. We herein developed a metapopulation dynamics model for semisessile highly-exploited species. Four models were developed considering different the habitat fragmentation, isolation and distance of populations in two oceanic archipelagos. We used as empirical example, the populations of two harvested intertidal limpets (\textit{Patella candei} and \textit{P. crenata}) in Madeira, and Tenerife (Canary Islands), two Atlantic volcanic islands distant to the continent.

\section{Keywords}
Metacommunities, patch dynamics, dispersal dynamics, individual-based model, Approximate Bayesian Computing

\section{Introduction}
Metapopulation models predict the survival of highly-disperse populations regardless local depletion of species (Hanski, 1999; Akçakaya et al. 2007). However, habitat loss and fragmentation may be so extensive that result in a massive species extinction (Montoya, 2008; Rybicki & Hanksi, 2013; Haddad et al. 2015). Model predictions have shown that a decrease in connectivity among assemblages, from continuous to sparsely-distributed populations is accompanied by species loss (Metzger et al. 2009; Niebuhr et al. 2015). The role of natural (e.g. stochastic events) and human-driven (e.g. pollution, harvesting, overfishing) perturbations have been extensively studied in the last decades (Dornelas, 2010 and references therein) and it has been frequently studied in ecological theory (e.g. Volkov et al. 2007; Gardner & Engelhardt, 2008). The degree of anthropogenic pressure may be a capital factor for landscape connectivity, since directly affects the persistence or decrease of assemblages (Supp & Ernest, 2014). Besides perturbations, dispersal rates in disturbed landscapes need to be high in order to maintain viable populations (Provan et al. 2009). However, individual-based models (IBM) are needed to predict dispersal rates in species where small specimens, i.e. juveniles, are not reproductively active and even larger-sized adults harbour the highest reproductive potential (e.g. Hendricks & Mulder, 2008; Werner & Griebeler, 2011). 
IBM simulate populations as being composed of discrete individual organisms (De Angelis & Grimm, 2014; Van der Väart et al. 2016). In IBMs the actions of single individuals are simulated and they interact with other and the landscape they live in (De Angelis & Mooij, 2005). They incorporate attributes vary among the individuals and can change through time such as, growth, foraging, dispersal and reproduction, among others (Martin et al. 2013; De Angelis & Grimm, 2014). These models have been used as size-structured methods to integrate a high variety of data which output are pivotal for management purposes (Punt et al. 2013) and conservation strategies (Nabe-Nielsen et al. 2014). Metapopulation models have been primarily used to certain vagile species such as insects (Harrison, 1995; Hilker et al. 2006) or mammals (Khrone, 1997; Lawes, 2000), but sessile or semi-sessile species have been neglected though a high proportion have a larval dispersal phase (Sale et al 2006; Gaggiotti, 2017). This strategy is extensively used in most of reduced-mobility marine species, with a subsequent increase of populations connectivity to nearby areas. However, assemblages from distant places are prone to isolation, as genetic analysis showed in semisessile mollusks from oceanic archipelagos (Corte-Real et al. 1996; Bird et al. 2011; Faria et al. 2017). Oceanographic conditions, i.e. currents and eddies, and large geographic distances showed to be pivotal environmental factors to decrease the dispersal capacity (e.g. Palumbi et al. 1994). High larvae mortality plays also a crucial role, mainly due to predation and larvae traits, i.e. growth, reproductive and recruitment strategies, and mobility capacity (Cowen et al. 2000; Cowen and Sponaugle 2009) in a limited connectivity within species. However, genetic analysis have been observed to be insufficient to accurately determine the demographic connectivity among populations of terrestrial (Chapuis et al. 2011), estuarine (Turner et al. 2002) and marine species (Hawkins et al. 2016).
We herein develop metacommunity models based on individuals of two intertidal species, namely the limpets \textit{Patella aspera} and \textit{P. candei}. The first model assumes that dispersal rates between patches are distance-dependent, with low rates between highly-separated assemblages. The second model assumes that dispersal rates are positively correlated to individual density. The third model assumes that larger individuals have higher reproductive potential. The fourth model considers a low probability of dispersal to peripheral assemblages relative to central ones. We confront the model with long-time series data (1994-2014) of two commercial limpet species (\textit{Patella candei} and \textit{P. aspera}) in two overpopulated oceanic islands (> 500 inhab km^{-2}) with a high harvesting pressure (see Riera et al. 2016; Sousa et al. 2018 for details). Former studies highlighted the sharp decrease of sizes of both limpets, a sympton of overexploitation and hence, a major driver for future local extinction of these species in Tenerife (Riera et al. 2016) and Madeira (Sousa et al. 2018). We herein develop a series of models  of metapopulation dynamics to predict future trends on the limpet populations of both islands. The predictive power of these models is pivotal to articulate an integrative management plan to preserve these endangered species.

\section{Material and Methods}\label

\underline{Material and methods where you have to include all methodology and procedures}

\subsection{Metapopulation model}
A metapopulation model was developed to explore the probability of occupancy of the two studied species in each oceanic island, i.e. Madeira and Tenerife, based on previous spatially-explicit metapopulation models (Hanski, 1999; Hanski and Ovaskainen, 2000; Ovaskainen and Hanski, 2001, 2002; Bertuzzo et al. 2015). Dispersal, recruitment and extinction are taken into account for the previous models. In the present model, each pixel of the modeled landscape is assumed to be a patch that may be either occupied or not by larvae of the studied limpet species. We assumed that the study species are constrained to disperse passively by the large-scale (the Canary Current, Barton et al. 1998) and meso-scale (eddies) oceanographic conditions,  and also by the active dispersion of the larvae (REFERENCES). Former studies have demonstrated that the main driver of the sharp decrease of limpet populations in both islands is the human harvesting pressure, we assumed that remains constant throughout the last 20 years, and with no seasonal variations within a year. Both species (\textit{Patella candei} and \textit{P. aspera}) are in clear regression in both islands due to their commercial interest (Riera et al. 2016; Sousa et al. 2018). We herein used the limpet size as a reproductive proxy, since individuals over 3.5 mm are considered adults, since 50\percent of individuals are reproductive (REFERENCE), and the higher the size the higher the reproductive potential (REFERENCE). Thus, a decrease of limpet size is a sympton of local extinctions in populations with low connectivity, as currently are those from isolated places such as, the oceanic islands of Madeira and Tenerife.



\section{Results}



\section{Discussion}



\section{Acknowledgements}




\section{References}
Akçakaya, H. R., Mills, G., & Doncaster, C. P. (2007). The role of metapopulations in conservation. Key topics in conservation biology, 64-84.
DeAngelis, DL. & V. Grimm. 2014. Individual-based models in ecology after four decades. F1000Prime Rep, 6: 39.
DeAngelis, DL. & WM. Mooij. 2005. Individual-based modelling of ecological and evolutionary processes. Annual Review of Ecology and Evolution Systematics, 36: 147-168.
Dornelas, M. 2010. Disturbance and change in biodiversity. Philosophical Transactions of the Royal Society B, 365: 3719-3727.
Gardner, RH. & KAM. Engelhardt. 2008. Spatial processes that maintain biodiversity in plant communities. Perspectives in Plant Ecology and Evolution Systematics, 9. 211-218
Haddad, N.M. et al.2015. Habitat fragmentation and its lasting impact on Earth´s ecosystems. Science Advances, 1: e1500052.
Hanski, I. 1999. Habitat connectivity, habitat continuity, and metapopulations in dynamic landscapes. Oikos, 87: 209-219.
Hendricks, AJ. & C. Mulder. 2008. Scaling of offspring number and mass to plant and animal size: model and meta-analysis. Oecologia, 155: 705-716.
Martin BT., T. Jager, RM. Nisbet, TG. Preuss, M. Hammers-Wirtz. & V. Grimm. 2013. Extrapolating ecotoxicological effects from individuals to populations: a generic approach based on Dynamic Energy Budget Theory and individual-based modelling. Ecotoxicology, 22: 574-583.
Montoya, D. 2008. Habitat loss, dispersal and the probability of extinction of tree species. Community Integrative Biology, 1: 146-147.
Nabe-Nielsen J., RM. Sibly, J. Tougaard, J. Teilmann. & S. Sveegard. 2014. Effects of noise and by-catch on a Danish harbor porpoise population. Ecological Modelling, 272: 242-251.
Niebuhr, BS, Wosniack, ME, Santos, MC. Raposo, EP, Viswanathan, GM. Da Luz, GE. & MR. Pie. 2015. Survival in patchy landscapes: the interplay between dispersal, habitat loss and fragmentation. Scientific Reports, 5: 11898.
Provan, J. GE. Beatty, SL. Keating, CA. Maggs & G. Savidge. 2009. High dispersal potential has maintained long-term population stability in the North Atlantic copepod Calanus finmarchicus. Proceedings of the Royal Society B, 276: 301-307.
Punt, AE, T. Huang & M.N. Maunder. 2013. Review of integrated size-structured models for stock assessment of hard-to-age crustacean and mollusk species. ICES Journal of Marine Science, 70: 16-33.
Riera R, Ó. Pérez, O. Álvarez, D. Simón, D. Díaz, Ó. Monterroso & J. Núñez. 2016. Clear regression of harvested intertidal mollusks. A 20-year (1994-2014) comparative study. Marine Environmental Research, 113: 56-61.
Supp, Sr. & SKM. Ernest. 2014. Species level and community level responses to disturbance: a cross-community analysis. Ecology, 95: 1717-1723.
Van de Väart, E. A. Johnston & RM. Sibly. 2016. Predicting how animals will de where: How to build, calibrate and evaluate individual-based models. Ecological Modelling, 326: 113-123.
Volkov, I. JR. Banavar, SP. Hubbell & A. Maritan. 2007. Patterns of relative species abundance in rainforests and coral reefs. Nature, 450: 45-49.
Werner, J. & EM, Griebeler. 2011. Reproductive biology and its impact on body size: comparative analysis of mammalian, avian and dinosaurian reproduction. PLoS ONE, 6: e28442.


\end{document}



